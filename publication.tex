%!TEX TS-program = xelatex
%!TEX encoding = UTF-8 Unicode
% Awesome CV LaTeX Template for CV/Resume
%
% This template has been downloaded from:
% https://github.com/posquit0/Awesome-CV
%
% Author:
% Claud D. Park <posquit0.bj@gmail.com>
% http://www.posquit0.com
%
%
% Adapted to be an Rmarkdown template by Mitchell O'Hara-Wild
% 23 November 2018
%
% Template license:
% CC BY-SA 4.0 (https://creativecommons.org/licenses/by-sa/4.0/)
%
%-------------------------------------------------------------------------------
% CONFIGURATIONS
%-------------------------------------------------------------------------------
% A4 paper size by default, use 'letterpaper' for US letter
\documentclass[11pt,a4paper,]{awesome-cv}

% Configure page margins with geometry
\usepackage{geometry}
\geometry{left=1.4cm, top=.8cm, right=1.4cm, bottom=1.8cm, footskip=.5cm}


% Specify the location of the included fonts
\fontdir[fonts/]

% Color for highlights
% Awesome Colors: awesome-emerald, awesome-skyblue, awesome-red, awesome-pink, awesome-orange
%                 awesome-nephritis, awesome-concrete, awesome-darknight

\colorlet{awesome}{awesome-red}

% Colors for text
% Uncomment if you would like to specify your own color
% \definecolor{darktext}{HTML}{414141}
% \definecolor{text}{HTML}{333333}
% \definecolor{graytext}{HTML}{5D5D5D}
% \definecolor{lighttext}{HTML}{999999}

% Set false if you don't want to highlight section with awesome color
\setbool{acvSectionColorHighlight}{true}

% If you would like to change the social information separator from a pipe (|) to something else
\renewcommand{\acvHeaderSocialSep}{\quad\textbar\quad}

\def\endfirstpage{\newpage}

%-------------------------------------------------------------------------------
%	PERSONAL INFORMATION
%	Comment any of the lines below if they are not required
%-------------------------------------------------------------------------------
% Available options: circle|rectangle,edge/noedge,left/right

\name{}{}



% \gitlab{gitlab-id}
% \stackoverflow{SO-id}{SO-name}
% \skype{skype-id}
% \reddit{reddit-id}


\usepackage{booktabs}

\providecommand{\tightlist}{%
	\setlength{\itemsep}{0pt}\setlength{\parskip}{0pt}}

%------------------------------------------------------------------------------

\usepackage{color}
\usepackage{fancyvrb}
\newcommand{\VerbBar}{|}
\newcommand{\VERB}{\Verb[commandchars=\\\{\}]}
\DefineVerbatimEnvironment{Highlighting}{Verbatim}{commandchars=\\\{\}}
% Add ',fontsize=\small' for more characters per line
\usepackage{framed}
\definecolor{shadecolor}{RGB}{248,248,248}
\newenvironment{Shaded}{\begin{snugshade}}{\end{snugshade}}
\newcommand{\AlertTok}[1]{\textcolor[rgb]{0.94,0.16,0.16}{#1}}
\newcommand{\AnnotationTok}[1]{\textcolor[rgb]{0.56,0.35,0.01}{\textbf{\textit{#1}}}}
\newcommand{\AttributeTok}[1]{\textcolor[rgb]{0.13,0.29,0.53}{#1}}
\newcommand{\BaseNTok}[1]{\textcolor[rgb]{0.00,0.00,0.81}{#1}}
\newcommand{\BuiltInTok}[1]{#1}
\newcommand{\CharTok}[1]{\textcolor[rgb]{0.31,0.60,0.02}{#1}}
\newcommand{\CommentTok}[1]{\textcolor[rgb]{0.56,0.35,0.01}{\textit{#1}}}
\newcommand{\CommentVarTok}[1]{\textcolor[rgb]{0.56,0.35,0.01}{\textbf{\textit{#1}}}}
\newcommand{\ConstantTok}[1]{\textcolor[rgb]{0.56,0.35,0.01}{#1}}
\newcommand{\ControlFlowTok}[1]{\textcolor[rgb]{0.13,0.29,0.53}{\textbf{#1}}}
\newcommand{\DataTypeTok}[1]{\textcolor[rgb]{0.13,0.29,0.53}{#1}}
\newcommand{\DecValTok}[1]{\textcolor[rgb]{0.00,0.00,0.81}{#1}}
\newcommand{\DocumentationTok}[1]{\textcolor[rgb]{0.56,0.35,0.01}{\textbf{\textit{#1}}}}
\newcommand{\ErrorTok}[1]{\textcolor[rgb]{0.64,0.00,0.00}{\textbf{#1}}}
\newcommand{\ExtensionTok}[1]{#1}
\newcommand{\FloatTok}[1]{\textcolor[rgb]{0.00,0.00,0.81}{#1}}
\newcommand{\FunctionTok}[1]{\textcolor[rgb]{0.13,0.29,0.53}{\textbf{#1}}}
\newcommand{\ImportTok}[1]{#1}
\newcommand{\InformationTok}[1]{\textcolor[rgb]{0.56,0.35,0.01}{\textbf{\textit{#1}}}}
\newcommand{\KeywordTok}[1]{\textcolor[rgb]{0.13,0.29,0.53}{\textbf{#1}}}
\newcommand{\NormalTok}[1]{#1}
\newcommand{\OperatorTok}[1]{\textcolor[rgb]{0.81,0.36,0.00}{\textbf{#1}}}
\newcommand{\OtherTok}[1]{\textcolor[rgb]{0.56,0.35,0.01}{#1}}
\newcommand{\PreprocessorTok}[1]{\textcolor[rgb]{0.56,0.35,0.01}{\textit{#1}}}
\newcommand{\RegionMarkerTok}[1]{#1}
\newcommand{\SpecialCharTok}[1]{\textcolor[rgb]{0.81,0.36,0.00}{\textbf{#1}}}
\newcommand{\SpecialStringTok}[1]{\textcolor[rgb]{0.31,0.60,0.02}{#1}}
\newcommand{\StringTok}[1]{\textcolor[rgb]{0.31,0.60,0.02}{#1}}
\newcommand{\VariableTok}[1]{\textcolor[rgb]{0.00,0.00,0.00}{#1}}
\newcommand{\VerbatimStringTok}[1]{\textcolor[rgb]{0.31,0.60,0.02}{#1}}
\newcommand{\WarningTok}[1]{\textcolor[rgb]{0.56,0.35,0.01}{\textbf{\textit{#1}}}}


% Pandoc CSL macros
% definitions for citeproc citations
\NewDocumentCommand\citeproctext{}{}
\NewDocumentCommand\citeproc{mm}{%
  \begingroup\def\citeproctext{#2}\cite{#1}\endgroup}
\makeatletter
 % allow citations to break across lines
 \let\@cite@ofmt\@firstofone
 % avoid brackets around text for \cite:
 \def\@biblabel#1{}
 \def\@cite#1#2{{#1\if@tempswa , #2\fi}}
\makeatother
\newlength{\cslhangindent}
\setlength{\cslhangindent}{1.5em}
\newlength{\csllabelwidth}
\setlength{\csllabelwidth}{3em}
\newenvironment{CSLReferences}[2] % #1 hanging-indent, #2 entry-spacing
 {\begin{list}{}{%
  \setlength{\itemindent}{0pt}
  \setlength{\leftmargin}{0pt}
  \setlength{\parsep}{0pt}
  % turn on hanging indent if param 1 is 1
  \ifodd #1
   \setlength{\leftmargin}{\cslhangindent}
   \setlength{\itemindent}{-1\cslhangindent}
  \fi
  % set entry spacing
  \setlength{\itemsep}{#2\baselineskip}}}
 {\end{list}}

\usepackage{calc}
\newcommand{\CSLBlock}[1]{\hfill\break\parbox[t]{\linewidth}{\strut\ignorespaces#1\strut}}
\newcommand{\CSLLeftMargin}[1]{\parbox[t]{\csllabelwidth}{\strut#1\strut}}
\newcommand{\CSLRightInline}[1]{\parbox[t]{\linewidth - \csllabelwidth}{\strut#1\strut}}
\newcommand{\CSLIndent}[1]{\hspace{\cslhangindent}#1}

\begin{document}

% Print the header with above personal informations
% Give optional argument to change alignment(C: center, L: left, R: right)
\makecvheader

% Print the footer with 3 arguments(<left>, <center>, <right>)
% Leave any of these blank if they are not needed
% 2019-02-14 Chris Umphlett - add flexibility to the document name in footer, rather than have it be static Curriculum Vitae
\makecvfooter
  {}
    {~~~·~~~Curriculum Vitae}
  {\thepage~ of \pageref{LastPage}~}


%-------------------------------------------------------------------------------
%	CV/RESUME CONTENT
%	Each section is imported separately, open each file in turn to modify content
%------------------------------------------------------------------------------



\section{Publications}\label{publications}

\textbf{\texttt{\#}}: Author of Correspondence.      
\textbf{\texttt{*}}: Joint First authors.       \textbf{\texttt{-}}: Not
Peer Reviewed.

\begin{Shaded}
\begin{Highlighting}[]
\FunctionTok{library}\NormalTok{(dplyr)}
\end{Highlighting}
\end{Shaded}

\begin{verbatim}
## 
## Attaching package: 'dplyr'
\end{verbatim}

\begin{verbatim}
## The following objects are masked from 'package:stats':
## 
##     filter, lag
\end{verbatim}

\begin{verbatim}
## The following objects are masked from 'package:base':
## 
##     intersect, setdiff, setequal, union
\end{verbatim}

\begin{Shaded}
\begin{Highlighting}[]
\FunctionTok{library}\NormalTok{(vitae)}
\end{Highlighting}
\end{Shaded}

\begin{verbatim}
## 
## Attaching package: 'vitae'
\end{verbatim}

\begin{verbatim}
## The following object is masked from 'package:stats':
## 
##     filter
\end{verbatim}

\begin{Shaded}
\begin{Highlighting}[]
\FunctionTok{library}\NormalTok{(tidyverse)}
\end{Highlighting}
\end{Shaded}

\begin{verbatim}
## -- Attaching core tidyverse packages ------------------------ tidyverse 2.0.0 --
## v forcats   1.0.0     v readr     2.1.5
## v ggplot2   3.5.1     v stringr   1.5.1
## v lubridate 1.9.3     v tibble    3.2.1
## v purrr     1.0.2     v tidyr     1.3.1
\end{verbatim}

\begin{verbatim}
## -- Conflicts ------------------------------------------ tidyverse_conflicts() --
## x vitae::filter() masks dplyr::filter(), stats::filter()
## x dplyr::lag()    masks stats::lag()
## i Use the conflicted package (<http://conflicted.r-lib.org/>) to force all conflicts to become errors
\end{verbatim}

\begin{Shaded}
\begin{Highlighting}[]
\FunctionTok{library}\NormalTok{(vroom)}
\end{Highlighting}
\end{Shaded}

\begin{verbatim}
## 
## Attaching package: 'vroom'
## 
## The following objects are masked from 'package:readr':
## 
##     as.col_spec, col_character, col_date, col_datetime, col_double,
##     col_factor, col_guess, col_integer, col_logical, col_number,
##     col_skip, col_time, cols, cols_condense, cols_only, date_names,
##     date_names_lang, date_names_langs, default_locale, fwf_cols,
##     fwf_empty, fwf_positions, fwf_widths, locale, output_column,
##     problems, spec
\end{verbatim}

\begin{Shaded}
\begin{Highlighting}[]
\FunctionTok{bibliography\_entries}\NormalTok{(}\StringTok{"data/pubs.bib"}\NormalTok{) }\SpecialCharTok{\%\textgreater{}\%}
    \FunctionTok{arrange}\NormalTok{(}\FunctionTok{desc}\NormalTok{(issued))}
\end{Highlighting}
\end{Shaded}

\phantomsection\label{refs-1797e7d470caf82193c45094f2f30c28}
\begin{CSLReferences}{0}{0}
\bibitem[\citeproctext]{ref-mathis2024self}
\CSLLeftMargin{1. }%
\CSLRightInline{Mathis, C., Patel, D., Weimer, W., \& Forrest, S.
(2024). SELF ORGANIZATION IN COMPUTATION \& CHEMISTRY: RETURN TO
ALCHEMY. \emph{Under Review}.
\CSLBlock{-}}

\bibitem[\citeproctext]{ref-foote2023false}
\CSLLeftMargin{2. }%
\CSLRightInline{Foote, S., Sinhadc, P., Mathis, C., \& Walker, S. I.
(2023). False positives and the challenge of testing the alien
hypothesis. \emph{Astrobiology}, \emph{23}(11), 1189--1201.
\CSLBlock{\#}}

\bibitem[\citeproctext]{ref-jirasek2023multimodal}
\CSLLeftMargin{3. }%
\CSLRightInline{Jirasek, M., Sharma, A., Bame, J. R., Bell, N.,
Marshall, S. M., Mathis, C., Macleod, A., Cooper, G. J., Swart, M.,
Mollfulleda, R., et al. (2023). Multimodal techniques for detecting
alien life using assembly theory and spectroscopy. \emph{arXiv Preprint
arXiv:2302.13753}.
\CSLBlock{-}}

\bibitem[\citeproctext]{ref-oolen2023takes}
\CSLLeftMargin{4. }%
\CSLRightInline{OoLEN, Asche, S., Bautista, C., Boulesteix, D.,
Champagne-Ruel, A., Mathis, C., Markovitch, O., Peng, Z., Adams, A.,
Dass, A. V., et al. (2023). What it takes to solve the origin (s) of
life: An integrated review of techniques. \emph{arXiv Preprint
arXiv:2308.11665}.
\CSLBlock{-}}

\bibitem[\citeproctext]{ref-smith2023life}
\CSLLeftMargin{5. }%
\CSLRightInline{Smith, H. B., \& Mathis, C. (2023). Life detection in a
universe of false positives: Can the fatal flaws of exoplanet
biosignatures be overcome absent a theory of life? \emph{BioEssays},
\emph{45}(12), 2300050.
\CSLBlock{*}}

\bibitem[\citeproctext]{ref-lockey2022investigating}
\CSLLeftMargin{6. }%
\CSLRightInline{Lockey, D., Mathis, C., Miras, H. N., \& Cronin, L.
(2022). Investigating the autocatalytically driven formation of
keggin-based polyoxometalate clusters. \emph{Matter}, \emph{5}(1),
302--313.}

\bibitem[\citeproctext]{ref-meadows2022community}
\CSLLeftMargin{7. }%
\CSLRightInline{Meadows, V., Graham, H., Abrahamsson, V., Adam, Z.,
Amador-French, E., Arney, G., Barge, L., Barlow, E., Berea, A., Bose,
M., et al. (2022). Community report from the biosignatures standards of
evidence workshop. \emph{arXiv Preprint arXiv:2210.14293}.
\CSLBlock{-}}

\bibitem[\citeproctext]{ref-mathis2020identifying}
\CSLLeftMargin{8. }%
\CSLRightInline{Mathis, C., Marshall, S., Carrick, E., Keenan, G.,
Cooper, G., Graham, H., Bame, J., Craven, M., Bell, N., Gromski, P. S.,
et al. (2021). Identifying molecules as biosignatures with assembly
theory and mass spectrometry. \emph{Nature Communications},
\emph{12}(1), 3033.
\CSLBlock{*}}

\bibitem[\citeproctext]{ref-liu2021exploring}
\CSLLeftMargin{9. }%
\CSLRightInline{Liu, Y., Mathis, C., Bajczyk, M. D., Marshall, S. M.,
Wilbraham, L., \& Cronin, L. (2021). Exploring and mapping chemical
space with molecular assembly trees. \emph{Science Advances},
\emph{7}(39), eabj2465.}

\bibitem[\citeproctext]{ref-asche2021robotic}
\CSLLeftMargin{10. }%
\CSLRightInline{Asche, S., Cooper, G. J., Keenan, G., Mathis, C., \&
Cronin, L. (2021). A robotic prebiotic chemist probes long term
reactions of complexifying mixtures. \emph{Nature Communications},
\emph{12}(1), 3547.}

\bibitem[\citeproctext]{ref-doran2021exploring}
\CSLLeftMargin{11. }%
\CSLRightInline{Doran, D., Clarke, E., Keenan, G., Carrick, E., Mathis,
C., \& Cronin, L. (2021). Exploring the sequence space of unknown
oligomers and polymers. \emph{Cell Reports Physical Science},
\emph{2}(12).}

\bibitem[\citeproctext]{ref-miras2020spontaneous}
\CSLLeftMargin{12. }%
\CSLRightInline{Miras, H. N., Mathis, C., Xuan, W., Long, D.-L., Pow,
R., \& Cronin, L. (2020). Spontaneous formation of autocatalytic sets
with self-replicating inorganic metal oxide clusters. \emph{Proceedings
of the National Academy of Sciences}, \emph{117}(20), 10699--10705.}

\bibitem[\citeproctext]{ref-mathis2020meaning}
\CSLLeftMargin{13. }%
\CSLRightInline{Mathis, C. (2020). Meaning of the living state.
\emph{Social and Conceptual Issues in Astrobiology}, 91.}

\bibitem[\citeproctext]{ref-antonioni2019individual}
\CSLLeftMargin{14. }%
\CSLRightInline{Antonioni, A., Martinez-Vaquero, L. A., Mathis, C.,
Peel, L., \& Stella, M. (2019). Individual perception dynamics in drunk
games. \emph{Physical Review E}, \emph{99}(5), 052311.}

\bibitem[\citeproctext]{ref-kim2019universal}
\CSLLeftMargin{15. }%
\CSLRightInline{Kim, H., Smith, H. B., Mathis, C., Raymond, J., \&
Walker, S. I. (2019). Universal scaling across biochemical networks on
earth. \emph{Science Advances}, \emph{5}(1), eaau0149.}

\bibitem[\citeproctext]{ref-surman2019environmental}
\CSLLeftMargin{16. }%
\CSLRightInline{Surman, A. J., Rodriguez-Garcia, M., Abul-Haija, Y. M.,
Cooper, G. J., Gromski, P. S., Turk-MacLeod, R., Mullin, M., Mathis, C.,
Walker, S. I., \& Cronin, L. (2019). Environmental control programs the
emergence of distinct functional ensembles from unconstrained chemical
reactions. \emph{Proceedings of the National Academy of Sciences},
\emph{116}(12), 5387--5392.}

\bibitem[\citeproctext]{ref-mathis2019autocatalysis}
\CSLLeftMargin{17. }%
\CSLRightInline{Mathis, C., Miras, H., \& Cronin, L. (2019).
Autocatalysis in a hierarchically organized inorganic chemical network.
\emph{Artificial Life Conference Proceedings}, 652--653.
\CSLBlock{-}}

\bibitem[\citeproctext]{ref-walker2018network}
\CSLLeftMargin{18. }%
\CSLRightInline{Walker, S. I., \& Mathis, C. (2018). Network theory in
prebiotic evolution. \emph{Prebiotic Chemistry and Chemical Evolution of
Nucleic Acids}, 263--291.
\CSLBlock{-}}

\bibitem[\citeproctext]{ref-mathis2017emergence}
\CSLLeftMargin{19. }%
\CSLRightInline{Mathis, C., Bhattacharya, T., \& Walker, S. I. (2017).
The emergence of life as a first-order phase transition.
\emph{Astrobiology}, \emph{17}(3), 266--276.}

\bibitem[\citeproctext]{ref-mathis2017prebiotic}
\CSLLeftMargin{20. }%
\CSLRightInline{Mathis, C., Ramprasad, S. N., Walker, S. I., \& Lehman,
N. (2017). Prebiotic RNA network formation: A taxonomy of molecular
cooperation. \emph{Life}, \emph{7}(4), 38.}

\end{CSLReferences}


\label{LastPage}~
\end{document}
